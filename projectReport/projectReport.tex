\documentclass{article} % For LaTeX2e
\usepackage{nips13submit_e,times}
\usepackage{hyperref}
\usepackage{url}
%\documentstyle[nips13submit_09,times,art10]{article} % For LaTeX 2.09


\title{Unsupervised Feature Learning for Object Classification}


\author{
Laxman Dhulipala, Harry Gifford, Wangzi He \\
Department of Computer Science\\
Carnegie Mellon University University\\
Pittsburgh, PA 15213 \\
\texttt{\{ldhulipa, hgifford, wangzih\}@andrew.cmu.edu} \\
}

% The \author macro works with any number of authors. There are two commands
% used to separate the names and addresses of multiple authors: \And and \AND.
%
% Using \And between authors leaves it to \LaTeX{} to determine where to break
% the lines. Using \AND forces a linebreak at that point. So, if \LaTeX{}
% puts 3 of 4 authors names on the first line, and the last on the second
% line, try using \AND instead of \And before the third author name.

\newcommand{\fix}{\marginpar{FIX}}
\newcommand{\new}{\marginpar{NEW}}

\nipsfinalcopy % Uncomment for camera-ready version

\begin{document}


\maketitle

\begin{abstract}
A major theme in modern object recognition research has been the use of 
features to boost classification techniques. Stemming from the seminal work
of Viola and Jones, features and feature learning techniques have been used 
year after year to improve existing object recognition algorithms, and increase 
classification rates. In this work, we implement and evaluate an object classification framework 
that first learns a dictionary of features, and subsequently uses this feature basis 
to represent and classify images into categories. We evaluate our classification framework
by analyzing its performance on the CIFAR-10 dataset, and also consider several optimizations
and heuristics to help boost performance. 
\end{abstract}

\section{Introduction}

Object Recognition as a field has its roots firmly embedded in a long history 
of research in Neuroscience. Stemming from Hubel and Wiesel's discovery in the 
late 1950's of complex and simple cells, researchers have been captivated by using
techniques and ideas from neuroscience as building blocks for robust object 
classification in a machine\cite{hubel}. To this end, ideas from neuroscience such as the Perceptron
algorithm, convolution neural networks and most recently, deep belief networks have all
had tremendous success in object recognition and machine learning. 

A fundamental step for many object recognition algorithms is to learn hierarchies
of features from images. Learnt features from a dataset afford the algorithm designer
to then represent images in the dataset as linear combinations of the learnt features. 
In some sense, this is the most natural reason to learn features, as they allow one 
to transform the space of images into the space of images as represented by their features.
This transformed space typically produces better results when running object classification
algorithms - even naive classification techniques such as $k$-nearest neighbors. 

Feature heirarchies can be learnt in a variety of ways, from semi-supervised learning
algorithms to fully unsupervised learning algorithms. 
Our approach to feature learning relies on an elegant and simple algorithm that was
recently published by Adam Coates and Andrew Ng \cite{coates}. They propose the use
of $k$-means clustering for feature learning as it is a simple and easily paralellizable
algorithm that can be deployed at scale. In an earlier work, they show how $k$-means in 
practice is highly competitive against the state-of-the-art feature learning algorithms, 
often beating them \cite{coates11}. In particular, they show a surprising result that 
$k$-means based feature learning with a single layer neural network achieves state-of-the-art
error rate on the CIFAR-10 dataset. 


\section{Related Work}

Object classification has had a long history of research and experimental results. In 
particular, many researchers in the past decade have made attempts at understanding the 
CIFAR-10 dataset with varying degrees of success. In this section we first describe
several other methods for performing feature learning, consider several algorithms which
use features to boost object classification performance, and finally describe the success
of some of these methods on the CIFAR-10 dataset. 

Feature learning can be split into three categories - supervised approaches
semi-supervised approaches, and lastly unsupervised feature learning. 
In terms of semi-supervised approaches, perhaps the most classic approach is that of 
Nigam et al. who describes how to 
learn text classifiers when there are limited number of labeled examples \cite{nigam}. 
They in a sense
overcome the dearth of labeled examples by using unlabeled documents to support the data.
This is a somewhat counterintuitive result, but can be rationalized by understanding that
unlabeled data, based on inferences about which class an unlabeled example belongs to can
be used to boost the joint probability distribution of features in the document. The authors
ultimately used EM (Expectation Maximization) with the assumption that the underlying data
arises from a mixture model to classify text. 

Another recent feature learning approach is pioneered by Raina et al. and runs under the 
moniker of Self-Taught Learning \cite{raina}. Self-Taught Learning works similarly to 
the EM-based 
approach described above, but makes no assumptions about the underlying distribution or
classes of the unsupervised data (there is no assumption that the unlabeled data even 
contains objects that are provided in the training and test datasets). Instead, they exploit
the fact that the unsupervised data is a collection of natural images. This makes their
algorithm significantly easier to apply in practice, as one can simply take a set of labeled
examples, and augment it with an enormous unlabeled data-set of support examples.

Both approaches are interesting due to their real world practicality - which we are deeply
concerned with. Both approaches augment their small set of labeled examples with a large 
number of unlabeled examples, and use the unlabeled examples to improve performance on the
test dataset. In the world, labeled examples are prohibitively costly -
therefore it is imperative to have techniques that will be almost fully unsupervised. 

\section{Headings: first level}
\label{headings}

First level headings are lower case (except for first word and proper nouns),
flush left, bold and in point size 12. One line space before the first level
heading and 1/2~line space after the first level heading.

\subsection{Headings: second level}

Second level headings are lower case (except for first word and proper nouns),
flush left, bold and in point size 10. One line space before the second level
heading and 1/2~line space after the second level heading.

\subsubsection{Headings: third level}

Third level headings are lower case (except for first word and proper nouns),
flush left, bold and in point size 10. One line space before the third level
heading and 1/2~line space after the third level heading.

\section{Citations, figures, tables, references}
\label{others}

These instructions apply to everyone, regardless of the formatter being used.

\subsection{Figures}

All artwork must be neat, clean, and legible. Lines should be dark
enough for purposes of reproduction; art work should not be
hand-drawn. The figure number and caption always appear after the
figure. Place one line space before the figure caption, and one line
space after the figure. The figure caption is lower case (except for
first word and proper nouns); figures are numbered consecutively.

Make sure the figure caption does not get separated from the figure.
Leave sufficient space to avoid splitting the figure and figure caption.

You may use color figures. 
However, it is best for the
figure captions and the paper body to make sense if the paper is printed
either in black/white or in color.
\begin{figure}[h]
\begin{center}
%\framebox[4.0in]{$\;$}
\fbox{\rule[-.5cm]{0cm}{4cm} \rule[-.5cm]{4cm}{0cm}}
\end{center}
\caption{Sample figure caption.}
\end{figure}

\subsection{Tables}

All tables must be centered, neat, clean and legible. Do not use hand-drawn
tables. The table number and title always appear before the table. See
Table~\ref{sample-table}.

Place one line space before the table title, one line space after the table
title, and one line space after the table. The table title must be lower case
(except for first word and proper nouns); tables are numbered consecutively.

\begin{table}[t]
\caption{Sample table title}
\label{sample-table}
\begin{center}
\begin{tabular}{ll}
\multicolumn{1}{c}{\bf PART}  &\multicolumn{1}{c}{\bf DESCRIPTION}
\\ \hline \\
Dendrite         &Input terminal \\
Axon             &Output terminal \\
Soma             &Cell body (contains cell nucleus) \\
\end{tabular}
\end{center}
\end{table}

\subsubsection*{}

\bibliographystyle{abbrv}
\bibliography{ref}

\end{document}
